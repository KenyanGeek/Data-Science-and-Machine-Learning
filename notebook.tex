
% Default to the notebook output style

    


% Inherit from the specified cell style.




    
\documentclass[11pt]{article}

    
    
    \usepackage[T1]{fontenc}
    % Nicer default font (+ math font) than Computer Modern for most use cases
    \usepackage{mathpazo}

    % Basic figure setup, for now with no caption control since it's done
    % automatically by Pandoc (which extracts ![](path) syntax from Markdown).
    \usepackage{graphicx}
    % We will generate all images so they have a width \maxwidth. This means
    % that they will get their normal width if they fit onto the page, but
    % are scaled down if they would overflow the margins.
    \makeatletter
    \def\maxwidth{\ifdim\Gin@nat@width>\linewidth\linewidth
    \else\Gin@nat@width\fi}
    \makeatother
    \let\Oldincludegraphics\includegraphics
    % Set max figure width to be 80% of text width, for now hardcoded.
    \renewcommand{\includegraphics}[1]{\Oldincludegraphics[width=.8\maxwidth]{#1}}
    % Ensure that by default, figures have no caption (until we provide a
    % proper Figure object with a Caption API and a way to capture that
    % in the conversion process - todo).
    \usepackage{caption}
    \DeclareCaptionLabelFormat{nolabel}{}
    \captionsetup{labelformat=nolabel}

    \usepackage{adjustbox} % Used to constrain images to a maximum size 
    \usepackage{xcolor} % Allow colors to be defined
    \usepackage{enumerate} % Needed for markdown enumerations to work
    \usepackage{geometry} % Used to adjust the document margins
    \usepackage{amsmath} % Equations
    \usepackage{amssymb} % Equations
    \usepackage{textcomp} % defines textquotesingle
    % Hack from http://tex.stackexchange.com/a/47451/13684:
    \AtBeginDocument{%
        \def\PYZsq{\textquotesingle}% Upright quotes in Pygmentized code
    }
    \usepackage{upquote} % Upright quotes for verbatim code
    \usepackage{eurosym} % defines \euro
    \usepackage[mathletters]{ucs} % Extended unicode (utf-8) support
    \usepackage[utf8x]{inputenc} % Allow utf-8 characters in the tex document
    \usepackage{fancyvrb} % verbatim replacement that allows latex
    \usepackage{grffile} % extends the file name processing of package graphics 
                         % to support a larger range 
    % The hyperref package gives us a pdf with properly built
    % internal navigation ('pdf bookmarks' for the table of contents,
    % internal cross-reference links, web links for URLs, etc.)
    \usepackage{hyperref}
    \usepackage{longtable} % longtable support required by pandoc >1.10
    \usepackage{booktabs}  % table support for pandoc > 1.12.2
    \usepackage[inline]{enumitem} % IRkernel/repr support (it uses the enumerate* environment)
    \usepackage[normalem]{ulem} % ulem is needed to support strikethroughs (\sout)
                                % normalem makes italics be italics, not underlines
    

    
    
    % Colors for the hyperref package
    \definecolor{urlcolor}{rgb}{0,.145,.698}
    \definecolor{linkcolor}{rgb}{.71,0.21,0.01}
    \definecolor{citecolor}{rgb}{.12,.54,.11}

    % ANSI colors
    \definecolor{ansi-black}{HTML}{3E424D}
    \definecolor{ansi-black-intense}{HTML}{282C36}
    \definecolor{ansi-red}{HTML}{E75C58}
    \definecolor{ansi-red-intense}{HTML}{B22B31}
    \definecolor{ansi-green}{HTML}{00A250}
    \definecolor{ansi-green-intense}{HTML}{007427}
    \definecolor{ansi-yellow}{HTML}{DDB62B}
    \definecolor{ansi-yellow-intense}{HTML}{B27D12}
    \definecolor{ansi-blue}{HTML}{208FFB}
    \definecolor{ansi-blue-intense}{HTML}{0065CA}
    \definecolor{ansi-magenta}{HTML}{D160C4}
    \definecolor{ansi-magenta-intense}{HTML}{A03196}
    \definecolor{ansi-cyan}{HTML}{60C6C8}
    \definecolor{ansi-cyan-intense}{HTML}{258F8F}
    \definecolor{ansi-white}{HTML}{C5C1B4}
    \definecolor{ansi-white-intense}{HTML}{A1A6B2}

    % commands and environments needed by pandoc snippets
    % extracted from the output of `pandoc -s`
    \providecommand{\tightlist}{%
      \setlength{\itemsep}{0pt}\setlength{\parskip}{0pt}}
    \DefineVerbatimEnvironment{Highlighting}{Verbatim}{commandchars=\\\{\}}
    % Add ',fontsize=\small' for more characters per line
    \newenvironment{Shaded}{}{}
    \newcommand{\KeywordTok}[1]{\textcolor[rgb]{0.00,0.44,0.13}{\textbf{{#1}}}}
    \newcommand{\DataTypeTok}[1]{\textcolor[rgb]{0.56,0.13,0.00}{{#1}}}
    \newcommand{\DecValTok}[1]{\textcolor[rgb]{0.25,0.63,0.44}{{#1}}}
    \newcommand{\BaseNTok}[1]{\textcolor[rgb]{0.25,0.63,0.44}{{#1}}}
    \newcommand{\FloatTok}[1]{\textcolor[rgb]{0.25,0.63,0.44}{{#1}}}
    \newcommand{\CharTok}[1]{\textcolor[rgb]{0.25,0.44,0.63}{{#1}}}
    \newcommand{\StringTok}[1]{\textcolor[rgb]{0.25,0.44,0.63}{{#1}}}
    \newcommand{\CommentTok}[1]{\textcolor[rgb]{0.38,0.63,0.69}{\textit{{#1}}}}
    \newcommand{\OtherTok}[1]{\textcolor[rgb]{0.00,0.44,0.13}{{#1}}}
    \newcommand{\AlertTok}[1]{\textcolor[rgb]{1.00,0.00,0.00}{\textbf{{#1}}}}
    \newcommand{\FunctionTok}[1]{\textcolor[rgb]{0.02,0.16,0.49}{{#1}}}
    \newcommand{\RegionMarkerTok}[1]{{#1}}
    \newcommand{\ErrorTok}[1]{\textcolor[rgb]{1.00,0.00,0.00}{\textbf{{#1}}}}
    \newcommand{\NormalTok}[1]{{#1}}
    
    % Additional commands for more recent versions of Pandoc
    \newcommand{\ConstantTok}[1]{\textcolor[rgb]{0.53,0.00,0.00}{{#1}}}
    \newcommand{\SpecialCharTok}[1]{\textcolor[rgb]{0.25,0.44,0.63}{{#1}}}
    \newcommand{\VerbatimStringTok}[1]{\textcolor[rgb]{0.25,0.44,0.63}{{#1}}}
    \newcommand{\SpecialStringTok}[1]{\textcolor[rgb]{0.73,0.40,0.53}{{#1}}}
    \newcommand{\ImportTok}[1]{{#1}}
    \newcommand{\DocumentationTok}[1]{\textcolor[rgb]{0.73,0.13,0.13}{\textit{{#1}}}}
    \newcommand{\AnnotationTok}[1]{\textcolor[rgb]{0.38,0.63,0.69}{\textbf{\textit{{#1}}}}}
    \newcommand{\CommentVarTok}[1]{\textcolor[rgb]{0.38,0.63,0.69}{\textbf{\textit{{#1}}}}}
    \newcommand{\VariableTok}[1]{\textcolor[rgb]{0.10,0.09,0.49}{{#1}}}
    \newcommand{\ControlFlowTok}[1]{\textcolor[rgb]{0.00,0.44,0.13}{\textbf{{#1}}}}
    \newcommand{\OperatorTok}[1]{\textcolor[rgb]{0.40,0.40,0.40}{{#1}}}
    \newcommand{\BuiltInTok}[1]{{#1}}
    \newcommand{\ExtensionTok}[1]{{#1}}
    \newcommand{\PreprocessorTok}[1]{\textcolor[rgb]{0.74,0.48,0.00}{{#1}}}
    \newcommand{\AttributeTok}[1]{\textcolor[rgb]{0.49,0.56,0.16}{{#1}}}
    \newcommand{\InformationTok}[1]{\textcolor[rgb]{0.38,0.63,0.69}{\textbf{\textit{{#1}}}}}
    \newcommand{\WarningTok}[1]{\textcolor[rgb]{0.38,0.63,0.69}{\textbf{\textit{{#1}}}}}
    
    
    % Define a nice break command that doesn't care if a line doesn't already
    % exist.
    \def\br{\hspace*{\fill} \\* }
    % Math Jax compatability definitions
    \def\gt{>}
    \def\lt{<}
    % Document parameters
    \title{Ongoing 7D Data Project}
    
    
    

    % Pygments definitions
    
\makeatletter
\def\PY@reset{\let\PY@it=\relax \let\PY@bf=\relax%
    \let\PY@ul=\relax \let\PY@tc=\relax%
    \let\PY@bc=\relax \let\PY@ff=\relax}
\def\PY@tok#1{\csname PY@tok@#1\endcsname}
\def\PY@toks#1+{\ifx\relax#1\empty\else%
    \PY@tok{#1}\expandafter\PY@toks\fi}
\def\PY@do#1{\PY@bc{\PY@tc{\PY@ul{%
    \PY@it{\PY@bf{\PY@ff{#1}}}}}}}
\def\PY#1#2{\PY@reset\PY@toks#1+\relax+\PY@do{#2}}

\expandafter\def\csname PY@tok@w\endcsname{\def\PY@tc##1{\textcolor[rgb]{0.73,0.73,0.73}{##1}}}
\expandafter\def\csname PY@tok@c\endcsname{\let\PY@it=\textit\def\PY@tc##1{\textcolor[rgb]{0.25,0.50,0.50}{##1}}}
\expandafter\def\csname PY@tok@cp\endcsname{\def\PY@tc##1{\textcolor[rgb]{0.74,0.48,0.00}{##1}}}
\expandafter\def\csname PY@tok@k\endcsname{\let\PY@bf=\textbf\def\PY@tc##1{\textcolor[rgb]{0.00,0.50,0.00}{##1}}}
\expandafter\def\csname PY@tok@kp\endcsname{\def\PY@tc##1{\textcolor[rgb]{0.00,0.50,0.00}{##1}}}
\expandafter\def\csname PY@tok@kt\endcsname{\def\PY@tc##1{\textcolor[rgb]{0.69,0.00,0.25}{##1}}}
\expandafter\def\csname PY@tok@o\endcsname{\def\PY@tc##1{\textcolor[rgb]{0.40,0.40,0.40}{##1}}}
\expandafter\def\csname PY@tok@ow\endcsname{\let\PY@bf=\textbf\def\PY@tc##1{\textcolor[rgb]{0.67,0.13,1.00}{##1}}}
\expandafter\def\csname PY@tok@nb\endcsname{\def\PY@tc##1{\textcolor[rgb]{0.00,0.50,0.00}{##1}}}
\expandafter\def\csname PY@tok@nf\endcsname{\def\PY@tc##1{\textcolor[rgb]{0.00,0.00,1.00}{##1}}}
\expandafter\def\csname PY@tok@nc\endcsname{\let\PY@bf=\textbf\def\PY@tc##1{\textcolor[rgb]{0.00,0.00,1.00}{##1}}}
\expandafter\def\csname PY@tok@nn\endcsname{\let\PY@bf=\textbf\def\PY@tc##1{\textcolor[rgb]{0.00,0.00,1.00}{##1}}}
\expandafter\def\csname PY@tok@ne\endcsname{\let\PY@bf=\textbf\def\PY@tc##1{\textcolor[rgb]{0.82,0.25,0.23}{##1}}}
\expandafter\def\csname PY@tok@nv\endcsname{\def\PY@tc##1{\textcolor[rgb]{0.10,0.09,0.49}{##1}}}
\expandafter\def\csname PY@tok@no\endcsname{\def\PY@tc##1{\textcolor[rgb]{0.53,0.00,0.00}{##1}}}
\expandafter\def\csname PY@tok@nl\endcsname{\def\PY@tc##1{\textcolor[rgb]{0.63,0.63,0.00}{##1}}}
\expandafter\def\csname PY@tok@ni\endcsname{\let\PY@bf=\textbf\def\PY@tc##1{\textcolor[rgb]{0.60,0.60,0.60}{##1}}}
\expandafter\def\csname PY@tok@na\endcsname{\def\PY@tc##1{\textcolor[rgb]{0.49,0.56,0.16}{##1}}}
\expandafter\def\csname PY@tok@nt\endcsname{\let\PY@bf=\textbf\def\PY@tc##1{\textcolor[rgb]{0.00,0.50,0.00}{##1}}}
\expandafter\def\csname PY@tok@nd\endcsname{\def\PY@tc##1{\textcolor[rgb]{0.67,0.13,1.00}{##1}}}
\expandafter\def\csname PY@tok@s\endcsname{\def\PY@tc##1{\textcolor[rgb]{0.73,0.13,0.13}{##1}}}
\expandafter\def\csname PY@tok@sd\endcsname{\let\PY@it=\textit\def\PY@tc##1{\textcolor[rgb]{0.73,0.13,0.13}{##1}}}
\expandafter\def\csname PY@tok@si\endcsname{\let\PY@bf=\textbf\def\PY@tc##1{\textcolor[rgb]{0.73,0.40,0.53}{##1}}}
\expandafter\def\csname PY@tok@se\endcsname{\let\PY@bf=\textbf\def\PY@tc##1{\textcolor[rgb]{0.73,0.40,0.13}{##1}}}
\expandafter\def\csname PY@tok@sr\endcsname{\def\PY@tc##1{\textcolor[rgb]{0.73,0.40,0.53}{##1}}}
\expandafter\def\csname PY@tok@ss\endcsname{\def\PY@tc##1{\textcolor[rgb]{0.10,0.09,0.49}{##1}}}
\expandafter\def\csname PY@tok@sx\endcsname{\def\PY@tc##1{\textcolor[rgb]{0.00,0.50,0.00}{##1}}}
\expandafter\def\csname PY@tok@m\endcsname{\def\PY@tc##1{\textcolor[rgb]{0.40,0.40,0.40}{##1}}}
\expandafter\def\csname PY@tok@gh\endcsname{\let\PY@bf=\textbf\def\PY@tc##1{\textcolor[rgb]{0.00,0.00,0.50}{##1}}}
\expandafter\def\csname PY@tok@gu\endcsname{\let\PY@bf=\textbf\def\PY@tc##1{\textcolor[rgb]{0.50,0.00,0.50}{##1}}}
\expandafter\def\csname PY@tok@gd\endcsname{\def\PY@tc##1{\textcolor[rgb]{0.63,0.00,0.00}{##1}}}
\expandafter\def\csname PY@tok@gi\endcsname{\def\PY@tc##1{\textcolor[rgb]{0.00,0.63,0.00}{##1}}}
\expandafter\def\csname PY@tok@gr\endcsname{\def\PY@tc##1{\textcolor[rgb]{1.00,0.00,0.00}{##1}}}
\expandafter\def\csname PY@tok@ge\endcsname{\let\PY@it=\textit}
\expandafter\def\csname PY@tok@gs\endcsname{\let\PY@bf=\textbf}
\expandafter\def\csname PY@tok@gp\endcsname{\let\PY@bf=\textbf\def\PY@tc##1{\textcolor[rgb]{0.00,0.00,0.50}{##1}}}
\expandafter\def\csname PY@tok@go\endcsname{\def\PY@tc##1{\textcolor[rgb]{0.53,0.53,0.53}{##1}}}
\expandafter\def\csname PY@tok@gt\endcsname{\def\PY@tc##1{\textcolor[rgb]{0.00,0.27,0.87}{##1}}}
\expandafter\def\csname PY@tok@err\endcsname{\def\PY@bc##1{\setlength{\fboxsep}{0pt}\fcolorbox[rgb]{1.00,0.00,0.00}{1,1,1}{\strut ##1}}}
\expandafter\def\csname PY@tok@kc\endcsname{\let\PY@bf=\textbf\def\PY@tc##1{\textcolor[rgb]{0.00,0.50,0.00}{##1}}}
\expandafter\def\csname PY@tok@kd\endcsname{\let\PY@bf=\textbf\def\PY@tc##1{\textcolor[rgb]{0.00,0.50,0.00}{##1}}}
\expandafter\def\csname PY@tok@kn\endcsname{\let\PY@bf=\textbf\def\PY@tc##1{\textcolor[rgb]{0.00,0.50,0.00}{##1}}}
\expandafter\def\csname PY@tok@kr\endcsname{\let\PY@bf=\textbf\def\PY@tc##1{\textcolor[rgb]{0.00,0.50,0.00}{##1}}}
\expandafter\def\csname PY@tok@bp\endcsname{\def\PY@tc##1{\textcolor[rgb]{0.00,0.50,0.00}{##1}}}
\expandafter\def\csname PY@tok@fm\endcsname{\def\PY@tc##1{\textcolor[rgb]{0.00,0.00,1.00}{##1}}}
\expandafter\def\csname PY@tok@vc\endcsname{\def\PY@tc##1{\textcolor[rgb]{0.10,0.09,0.49}{##1}}}
\expandafter\def\csname PY@tok@vg\endcsname{\def\PY@tc##1{\textcolor[rgb]{0.10,0.09,0.49}{##1}}}
\expandafter\def\csname PY@tok@vi\endcsname{\def\PY@tc##1{\textcolor[rgb]{0.10,0.09,0.49}{##1}}}
\expandafter\def\csname PY@tok@vm\endcsname{\def\PY@tc##1{\textcolor[rgb]{0.10,0.09,0.49}{##1}}}
\expandafter\def\csname PY@tok@sa\endcsname{\def\PY@tc##1{\textcolor[rgb]{0.73,0.13,0.13}{##1}}}
\expandafter\def\csname PY@tok@sb\endcsname{\def\PY@tc##1{\textcolor[rgb]{0.73,0.13,0.13}{##1}}}
\expandafter\def\csname PY@tok@sc\endcsname{\def\PY@tc##1{\textcolor[rgb]{0.73,0.13,0.13}{##1}}}
\expandafter\def\csname PY@tok@dl\endcsname{\def\PY@tc##1{\textcolor[rgb]{0.73,0.13,0.13}{##1}}}
\expandafter\def\csname PY@tok@s2\endcsname{\def\PY@tc##1{\textcolor[rgb]{0.73,0.13,0.13}{##1}}}
\expandafter\def\csname PY@tok@sh\endcsname{\def\PY@tc##1{\textcolor[rgb]{0.73,0.13,0.13}{##1}}}
\expandafter\def\csname PY@tok@s1\endcsname{\def\PY@tc##1{\textcolor[rgb]{0.73,0.13,0.13}{##1}}}
\expandafter\def\csname PY@tok@mb\endcsname{\def\PY@tc##1{\textcolor[rgb]{0.40,0.40,0.40}{##1}}}
\expandafter\def\csname PY@tok@mf\endcsname{\def\PY@tc##1{\textcolor[rgb]{0.40,0.40,0.40}{##1}}}
\expandafter\def\csname PY@tok@mh\endcsname{\def\PY@tc##1{\textcolor[rgb]{0.40,0.40,0.40}{##1}}}
\expandafter\def\csname PY@tok@mi\endcsname{\def\PY@tc##1{\textcolor[rgb]{0.40,0.40,0.40}{##1}}}
\expandafter\def\csname PY@tok@il\endcsname{\def\PY@tc##1{\textcolor[rgb]{0.40,0.40,0.40}{##1}}}
\expandafter\def\csname PY@tok@mo\endcsname{\def\PY@tc##1{\textcolor[rgb]{0.40,0.40,0.40}{##1}}}
\expandafter\def\csname PY@tok@ch\endcsname{\let\PY@it=\textit\def\PY@tc##1{\textcolor[rgb]{0.25,0.50,0.50}{##1}}}
\expandafter\def\csname PY@tok@cm\endcsname{\let\PY@it=\textit\def\PY@tc##1{\textcolor[rgb]{0.25,0.50,0.50}{##1}}}
\expandafter\def\csname PY@tok@cpf\endcsname{\let\PY@it=\textit\def\PY@tc##1{\textcolor[rgb]{0.25,0.50,0.50}{##1}}}
\expandafter\def\csname PY@tok@c1\endcsname{\let\PY@it=\textit\def\PY@tc##1{\textcolor[rgb]{0.25,0.50,0.50}{##1}}}
\expandafter\def\csname PY@tok@cs\endcsname{\let\PY@it=\textit\def\PY@tc##1{\textcolor[rgb]{0.25,0.50,0.50}{##1}}}

\def\PYZbs{\char`\\}
\def\PYZus{\char`\_}
\def\PYZob{\char`\{}
\def\PYZcb{\char`\}}
\def\PYZca{\char`\^}
\def\PYZam{\char`\&}
\def\PYZlt{\char`\<}
\def\PYZgt{\char`\>}
\def\PYZsh{\char`\#}
\def\PYZpc{\char`\%}
\def\PYZdl{\char`\$}
\def\PYZhy{\char`\-}
\def\PYZsq{\char`\'}
\def\PYZdq{\char`\"}
\def\PYZti{\char`\~}
% for compatibility with earlier versions
\def\PYZat{@}
\def\PYZlb{[}
\def\PYZrb{]}
\makeatother


    % Exact colors from NB
    \definecolor{incolor}{rgb}{0.0, 0.0, 0.5}
    \definecolor{outcolor}{rgb}{0.545, 0.0, 0.0}



    
    % Prevent overflowing lines due to hard-to-break entities
    \sloppy 
    % Setup hyperref package
    \hypersetup{
      breaklinks=true,  % so long urls are correctly broken across lines
      colorlinks=true,
      urlcolor=urlcolor,
      linkcolor=linkcolor,
      citecolor=citecolor,
      }
    % Slightly bigger margins than the latex defaults
    
    \geometry{verbose,tmargin=1in,bmargin=1in,lmargin=1in,rmargin=1in}
    
    

    \begin{document}
    
    
    \maketitle
    
    

    
    Table of Contents{}

{{1~~}INTRODUCTION}

{{1.1~~}BACKGROUND}

{{1.2~~}OBJECTIVES}

{{1.3~~}METHODOLOGY}

{{1.4~~}LIBRARIES}

{{2~~}COMBINE EXCEL DATA SHEETS}

{{2.1~~}Importing necessary libraries}

{{2.2~~}Combining the sheets}

{{3~~}IMPORT DATA TO PYTHON}

{{3.1~~}Importing Data from an Excel sheet}

{{3.2~~}Data Engineering}

    \hypertarget{introduction}{%
\section{INTRODUCTION}\label{introduction}}

    \hypertarget{background}{%
\subsection{BACKGROUND}\label{background}}

    I started the Nikon 7D business in 2014 in an upcountry town of Kenya. I
purposed to keep daily retail sales data for the past five years until i
sold it in March 2019 for a hefty profit. In this notebook; I use this
real life retail data to see how my business had being doing and utilize
open source data science tools to understand retail dynamics. Further am
interested in applying machine learning models to make predictions from
the collected data. Why? Just for fun - but anybody can follow this
workflow to analyse their business and make useful predictions.

    \hypertarget{objectives}{%
\subsection{OBJECTIVES}\label{objectives}}

    \begin{enumerate}
\def\labelenumi{\arabic{enumi}.}
\tightlist
\item
  Understand how much money i made for the whole period, what was the
  expenses and gross profit
\item
  Understand how much money i was making averagely every year, quarter,
  month, week, day
\item
  Which day of the week was i making the most money -- (so that if i was
  still running the business i would know when to increase resources)
\item
  Which day of the week was i making the least money -- (so that if i
  was still running the business i would know when to reduce resource
  expenses)
\item
  Which Month did i make the most money -- increase resources for
  delivery
\item
  Which Month made the least money -- (so that if i was still running
  the business i would know when increase marketing)
\item
  Remove 2019 data and do predictions with different models Fast.ai,
  Pytorch and Prophet and compare the efficability of the various models
\end{enumerate}

    \hypertarget{methodology}{%
\subsection{METHODOLOGY}\label{methodology}}

    \begin{enumerate}
\def\labelenumi{\arabic{enumi}.}
\tightlist
\item
  Data import from excel

  \begin{enumerate}
  \def\labelenumii{\arabic{enumii}.}
  \tightlist
  \item
    Combine all the sheets using excel or python scripts - Done with
    python and VB Macros
  \item
    Import that data using pandas excel functions
  \end{enumerate}
\item
  Data preparation and Munging

  \begin{enumerate}
  \def\labelenumii{\arabic{enumii}.}
  \tightlist
  \item
    Make more variabes from the available data using python
  \item
    Preliminary visualization of the data to help eliminate outliers
  \end{enumerate}
\item
  Data Visualization
\item
  Data Analysis to answer above questions
\item
  Machine learning for prediction of Tabular data using Fast.ai
\item
  Machine learning for prediction of Tabular data using Prophet
\item
  Machine learning for prediction of Tabular data using Pytorch
\item
  Machine learning for prediction of Tabular data using XGBOOST
\item
  Comparison of the best machine learning model
\end{enumerate}

    \hypertarget{libraries}{%
\subsection{LIBRARIES}\label{libraries}}

    I will use the following libraries and open source softwares: * General
Open source Distributions for Data Science - Ananconda Distribution -
Python * Python Data Analysis Libraries - Pandas * Python Visualization
Libraries - Matplotlib - Seaborn * Python Machine Learning Libraries -
Scikit Learn - XGBOOST Algorithms - Prophet - Fast.ai - Pytorch

    \hypertarget{combine-excel-data-sheets}{%
\section{COMBINE EXCEL DATA SHEETS}\label{combine-excel-data-sheets}}

    I have 43 excel sheets in one workbook representing the 43 months i had
collected data for the business, so in this section i will collate the
43 different sheets in an excel workbook to one file which i can then
work with.

    \hypertarget{importing-necessary-libraries}{%
\subsection{Importing necessary
libraries}\label{importing-necessary-libraries}}

    \begin{Verbatim}[commandchars=\\\{\}]
{\color{incolor}In [{\color{incolor}1}]:} \PY{c+c1}{\PYZsh{} Here i will import the following libraries }
        
        \PY{c+c1}{\PYZsh{}pandas — (Python for Data Analysis Library) — has libraries that help you carry out data analysis}
        \PY{c+c1}{\PYZsh{}numpy — (A library for playing around with numbers in Python)}
        \PY{c+c1}{\PYZsh{}os — helps you dip your hands into the operating system of whatever computer you are using}
        \PY{c+c1}{\PYZsh{}collections — are containers that are used to store collections of data}
        \PY{c+c1}{\PYZsh{}csv — library that reads and writes tabular data in CSV format.}
        \PY{c+c1}{\PYZsh{}basename — function found in the path method of the os library that extracts the base name or directory name conveniently}
        
        
        \PY{k+kn}{import} \PY{n+nn}{pandas} \PY{k}{as} \PY{n+nn}{pd} 
        \PY{k+kn}{import} \PY{n+nn}{numpy} \PY{k}{as} \PY{n+nn}{np}
        \PY{k+kn}{import} \PY{n+nn}{os}\PY{o}{,} \PY{n+nn}{collections}\PY{o}{,} \PY{n+nn}{csv}
        \PY{k+kn}{from} \PY{n+nn}{os}\PY{n+nn}{.}\PY{n+nn}{path}  \PY{k}{import} \PY{n}{basename}
\end{Verbatim}


    \hypertarget{combining-the-sheets}{%
\subsection{Combining the sheets}\label{combining-the-sheets}}

    \begin{Verbatim}[commandchars=\\\{\}]
{\color{incolor}In [{\color{incolor}2}]:} \PY{c+c1}{\PYZsh{} next create an empty dictionary with a variable name df}
        \PY{c+c1}{\PYZsh{} initialize f which is a variable that is assigned the name of the original workbook with the many sheets that you want to combine}
        \PY{c+c1}{\PYZsh{} store a variable numberOfSheets with the sheets you want to combine }
        \PY{n}{df} \PY{o}{=} \PY{p}{[}\PY{p}{]}
        
        
        \PY{n}{f} \PY{o}{=} \PY{l+s+s2}{\PYZdq{}}\PY{l+s+s2}{C:/Users/Njoroge Wa Chege/Desktop/7D DATA PROJECT/python test/income.xlsx}\PY{l+s+s2}{\PYZdq{}}
        \PY{n}{numberOfSheets} \PY{o}{=} \PY{l+m+mi}{43}
        
        \PY{c+c1}{\PYZsh{}you can use VBScripts to rename your excel sheets to have consistent names}
        
        \PY{k}{for} \PY{n}{i} \PY{o+ow}{in} \PY{n+nb}{range}\PY{p}{(}\PY{l+m+mi}{1}\PY{p}{,}\PY{n}{numberOfSheets}\PY{o}{+}\PY{l+m+mi}{1}\PY{p}{)}\PY{p}{:}    
             \PY{n}{data} \PY{o}{=} \PY{n}{pd}\PY{o}{.}\PY{n}{read\PYZus{}excel}\PY{p}{(}\PY{n}{f}\PY{p}{,} \PY{n}{sheet\PYZus{}name} \PY{o}{=} \PY{l+s+s1}{\PYZsq{}}\PY{l+s+s1}{month}\PY{l+s+s1}{\PYZsq{}}\PY{o}{+}\PY{n+nb}{str}\PY{p}{(}\PY{n}{i}\PY{p}{)}\PY{p}{,} \PY{n}{header}\PY{o}{=}\PY{k+kc}{None}\PY{p}{)} 
             \PY{n}{df}\PY{o}{.}\PY{n}{append}\PY{p}{(}\PY{n}{data}\PY{p}{)}
            
        \PY{c+c1}{\PYZsh{}remember python is very strict on how you arrange your forward and back slashes so be aware of this: i am using windows here and hence the forward slashes}
        
        \PY{c+c1}{\PYZsh{}determine the path where the final compiled file will go}
        
        \PY{n}{final} \PY{o}{=} \PY{l+s+s2}{\PYZdq{}}\PY{l+s+s2}{C:/Users/Njoroge Wa Chege/Desktop/7D DATA PROJECT/python test/mergedfile.xlsx}\PY{l+s+s2}{\PYZdq{}}
        
        \PY{c+c1}{\PYZsh{} I then use the .concat() function from pandas to join/concatenate the sheets together into one}
        
        \PY{n}{df} \PY{o}{=} \PY{n}{pd}\PY{o}{.}\PY{n}{concat}\PY{p}{(}\PY{n}{df}\PY{p}{)}
        
        \PY{c+c1}{\PYZsh{} I finally write the final combined excel sheet to the path i had defined in final!}
        
        \PY{n}{df}\PY{o}{.}\PY{n}{to\PYZus{}excel}\PY{p}{(}\PY{n}{final}\PY{p}{)}
\end{Verbatim}


    \hypertarget{import-data-to-python}{%
\section{IMPORT DATA TO PYTHON}\label{import-data-to-python}}

    \hypertarget{importing-data-from-an-excel-sheet}{%
\subsection{Importing Data from an Excel
sheet}\label{importing-data-from-an-excel-sheet}}

    \begin{Verbatim}[commandchars=\\\{\}]
{\color{incolor}In [{\color{incolor}3}]:} \PY{c+c1}{\PYZsh{} I will save the path to my file as a python variable:}
        \PY{c+c1}{\PYZsh{} Confirm that the file location has been saved}
        
        
        \PY{n}{mydata} \PY{o}{=}\PY{l+s+s1}{\PYZsq{}}\PY{l+s+s1}{C:/Users/Njoroge Wa Chege/Desktop/7D DATA PROJECT/python test/mergedfile.xlsx}\PY{l+s+s1}{\PYZsq{}}
        
        \PY{n}{mydata}
\end{Verbatim}


\begin{Verbatim}[commandchars=\\\{\}]
{\color{outcolor}Out[{\color{outcolor}3}]:} 'C:/Users/Njoroge Wa Chege/Desktop/7D DATA PROJECT/python test/mergedfile.xlsx'
\end{Verbatim}
            
    \begin{Verbatim}[commandchars=\\\{\}]
{\color{incolor}In [{\color{incolor}4}]:} \PY{c+c1}{\PYZsh{} I will then use pandas method .readexcel to read the data and convert it to a dataframe}
        \PY{c+c1}{\PYZsh{} I will then check the contents of my dataframe by calling the df variable}
        
        
        \PY{n}{df} \PY{o}{=} \PY{n}{pd}\PY{o}{.}\PY{n}{read\PYZus{}excel}\PY{p}{(} \PY{n}{mydata} \PY{p}{,} \PY{n}{sheet\PYZus{}name}\PY{o}{=}\PY{l+s+s1}{\PYZsq{}}\PY{l+s+s1}{Sheet1}\PY{l+s+s1}{\PYZsq{}}\PY{p}{)}
        
        
        \PY{n}{df}
\end{Verbatim}


\begin{Verbatim}[commandchars=\\\{\}]
{\color{outcolor}Out[{\color{outcolor}4}]:}             0                    1          2        3       4            5
        0         day                 date  netprofit  expense  bonus   grossprofit
        1      Monday  2015-09-07 00:00:00       2050      100       0         1950
        2     Tuesday  2015-09-08 00:00:00       1750       50       0         1700
        3   Wednesday  2015-09-09 00:00:00        700        0       0          700
        4    Thursday  2015-09-10 00:00:00       1100        0       0         1100
        5      Friday  2015-09-11 00:00:00       1100        0       0         1100
        6    Saturday  2015-09-12 00:00:00        750        0       0          750
        7      Sunday  2015-09-13 00:00:00       4600      200       0         4400
        8      Monday  2015-09-14 00:00:00       1250       50       0         1200
        9     Tuesday  2015-09-15 00:00:00       2150      100       0         2050
        10  Wednesday  2015-09-16 00:00:00       1700      500       0         1200
        11   Thursday  2015-09-17 00:00:00       4900      300       0         4600
        12     Friday  2015-09-18 00:00:00       1550       50       0         1500
        13   Saturday  2015-09-19 00:00:00        200        0       0          200
        14     Sunday  2015-09-20 00:00:00        950      100       0          850
        15     Monday  2015-09-21 00:00:00        600        0       0          600
        16    Tuesday  2015-09-22 00:00:00        550        0       0          550
        17  Wednesday  2015-09-23 00:00:00       1000        0       0         1000
        18   Thursday  2015-09-24 00:00:00        400        0       0          400
        19     Friday  2015-09-25 00:00:00        850        0       0          850
        20   Saturday  2015-09-26 00:00:00       1380      280       0         1100
        21     Sunday  2015-09-27 00:00:00        950      100       0          850
        22     Monday  2015-09-28 00:00:00       1765      265       0         1500
        23    Tuesday  2015-09-29 00:00:00        600        0       0          600
        24  Wednesday  2015-09-30 00:00:00       1550       50       0         1500
        25   Thursday  2015-10-01 00:00:00        600        0       0          600
        26     Friday  2015-10-02 00:00:00       1000        0       0         1000
        27   Saturday  2015-10-03 00:00:00        600        0       0          600
        28     Sunday  2015-10-04 00:00:00        900        0       0          900
        29     Monday  2015-10-05 00:00:00       2700      100       0         2600
        ..        {\ldots}                  {\ldots}        {\ldots}      {\ldots}     {\ldots}          {\ldots}
        7    Thursday  2019-02-14 00:00:00       1200        0     100         1100
        8      Friday  2019-02-15 00:00:00       1100        0     100         1000
        9    Saturday  2019-02-16 00:00:00       1250        0     100         1150
        10     Sunday  2019-02-17 00:00:00       1250       50     100         1100
        11     Monday  2019-02-18 00:00:00       1300        0     100         1200
        12    Tuesday  2019-02-19 00:00:00       1200       50     100         1050
        13  Wednesday  2019-02-20 00:00:00       1950      450     100         1400
        14   Thursday  2019-02-21 00:00:00       1100        0     100         1000
        15     Friday  2019-02-22 00:00:00       1250        0     100         1150
        16   Saturday  2019-02-23 00:00:00       5900      600     400         4900
        17     Sunday  2019-02-24 00:00:00       1250        0     100         1150
        18     Monday  2019-02-25 00:00:00       1100        0     100         1000
        19    Tuesday  2019-02-26 00:00:00       1400        0     100         1300
        20  Wednesday  2019-02-27 00:00:00        800        0     100          700
        21   Thursday  2019-02-28 00:00:00       1200        0     100         1100
        22     Friday  2019-03-01 00:00:00       1100        0     100         1000
        23   Saturday  2019-03-02 00:00:00       1400        0     100         1300
        24     Sunday  2019-03-03 00:00:00       1300       50     100         1150
        25     Monday  2019-03-04 00:00:00       1200      100     100         1000
        26    Tuesday  2019-03-05 00:00:00       1450       50     100         1300
        27  Wednesday  2019-03-06 00:00:00       1250      150     100         1000
        0    Thursday  2019-03-07 00:00:00       1100        0     100         1000
        1      Friday  2019-03-08 00:00:00       1500      500     100          900
        2    Saturday  2019-03-09 00:00:00       1350      200     100         1050
        3      Sunday  2019-03-10 00:00:00       1250        0     100         1150
        4      Monday  2019-03-11 00:00:00       1300       50     100         1150
        5     Tuesday  2019-03-12 00:00:00       1650       50     150         1450
        6   Wednesday  2019-03-13 00:00:00       1400      600     100          700
        7    Thursday  2019-03-14 00:00:00       1250        0     100         1150
        8      Friday  2019-03-15 00:00:00       1000        0     100          900
        
        [1288 rows x 6 columns]
\end{Verbatim}
            
    \hypertarget{data-engineering}{%
\subsection{Data Engineering}\label{data-engineering}}


    % Add a bibliography block to the postdoc
    
    
    
    \end{document}
